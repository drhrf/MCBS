% Options for packages loaded elsewhere
\PassOptionsToPackage{unicode}{hyperref}
\PassOptionsToPackage{hyphens}{url}
%
\documentclass[
]{book}
\usepackage{amsmath,amssymb}
\usepackage{iftex}
\ifPDFTeX
  \usepackage[T1]{fontenc}
  \usepackage[utf8]{inputenc}
  \usepackage{textcomp} % provide euro and other symbols
\else % if luatex or xetex
  \usepackage{unicode-math} % this also loads fontspec
  \defaultfontfeatures{Scale=MatchLowercase}
  \defaultfontfeatures[\rmfamily]{Ligatures=TeX,Scale=1}
\fi
\usepackage{lmodern}
\ifPDFTeX\else
  % xetex/luatex font selection
\fi
% Use upquote if available, for straight quotes in verbatim environments
\IfFileExists{upquote.sty}{\usepackage{upquote}}{}
\IfFileExists{microtype.sty}{% use microtype if available
  \usepackage[]{microtype}
  \UseMicrotypeSet[protrusion]{basicmath} % disable protrusion for tt fonts
}{}
\makeatletter
\@ifundefined{KOMAClassName}{% if non-KOMA class
  \IfFileExists{parskip.sty}{%
    \usepackage{parskip}
  }{% else
    \setlength{\parindent}{0pt}
    \setlength{\parskip}{6pt plus 2pt minus 1pt}}
}{% if KOMA class
  \KOMAoptions{parskip=half}}
\makeatother
\usepackage{xcolor}
\usepackage{longtable,booktabs,array}
\usepackage{calc} % for calculating minipage widths
% Correct order of tables after \paragraph or \subparagraph
\usepackage{etoolbox}
\makeatletter
\patchcmd\longtable{\par}{\if@noskipsec\mbox{}\fi\par}{}{}
\makeatother
% Allow footnotes in longtable head/foot
\IfFileExists{footnotehyper.sty}{\usepackage{footnotehyper}}{\usepackage{footnote}}
\makesavenoteenv{longtable}
\usepackage{graphicx}
\makeatletter
\def\maxwidth{\ifdim\Gin@nat@width>\linewidth\linewidth\else\Gin@nat@width\fi}
\def\maxheight{\ifdim\Gin@nat@height>\textheight\textheight\else\Gin@nat@height\fi}
\makeatother
% Scale images if necessary, so that they will not overflow the page
% margins by default, and it is still possible to overwrite the defaults
% using explicit options in \includegraphics[width, height, ...]{}
\setkeys{Gin}{width=\maxwidth,height=\maxheight,keepaspectratio}
% Set default figure placement to htbp
\makeatletter
\def\fps@figure{htbp}
\makeatother
\setlength{\emergencystretch}{3em} % prevent overfull lines
\providecommand{\tightlist}{%
  \setlength{\itemsep}{0pt}\setlength{\parskip}{0pt}}
\setcounter{secnumdepth}{5}
% definitions for citeproc citations
\NewDocumentCommand\citeproctext{}{}
\NewDocumentCommand\citeproc{mm}{%
  \begingroup\def\citeproctext{#2}\cite{#1}\endgroup}
\makeatletter
 % allow citations to break across lines
 \let\@cite@ofmt\@firstofone
 % avoid brackets around text for \cite:
 \def\@biblabel#1{}
 \def\@cite#1#2{{#1\if@tempswa , #2\fi}}
\makeatother
\newlength{\cslhangindent}
\setlength{\cslhangindent}{1.5em}
\newlength{\csllabelwidth}
\setlength{\csllabelwidth}{3em}
\newenvironment{CSLReferences}[2] % #1 hanging-indent, #2 entry-spacing
 {\begin{list}{}{%
  \setlength{\itemindent}{0pt}
  \setlength{\leftmargin}{0pt}
  \setlength{\parsep}{0pt}
  % turn on hanging indent if param 1 is 1
  \ifodd #1
   \setlength{\leftmargin}{\cslhangindent}
   \setlength{\itemindent}{-1\cslhangindent}
  \fi
  % set entry spacing
  \setlength{\itemsep}{#2\baselineskip}}}
 {\end{list}}
\usepackage{calc}
\newcommand{\CSLBlock}[1]{\hfill\break\parbox[t]{\linewidth}{\strut\ignorespaces#1\strut}}
\newcommand{\CSLLeftMargin}[1]{\parbox[t]{\csllabelwidth}{\strut#1\strut}}
\newcommand{\CSLRightInline}[1]{\parbox[t]{\linewidth - \csllabelwidth}{\strut#1\strut}}
\newcommand{\CSLIndent}[1]{\hspace{\cslhangindent}#1}
\usepackage{titlesec}

% Redefine "Chapter" to the desired term
\titleformat{\chapter}[display]
  {\normalfont\LARGE\bfseries}
  {\chaptername~\thechapter}{0.5em}{}
  
% Change the word "Chapter" to something else
\renewcommand{\chaptername}{Capítulo} % Substitua por qualquer palavra

% Redefine "Contents" to "Conteúdos"
\renewcommand{\contentsname}{Conteúdos}
\ifLuaTeX
  \usepackage{selnolig}  % disable illegal ligatures
\fi
\usepackage{bookmark}
\IfFileExists{xurl.sty}{\usepackage{xurl}}{} % add URL line breaks if available
\urlstyle{same}
\hypersetup{
  pdftitle={Metodologia científica e Bioestatística},
  pdfauthor={Editor: Hércules Rezende Freitas, Ph.D.},
  hidelinks,
  pdfcreator={LaTeX via pandoc}}

\title{Metodologia científica e Bioestatística}
\author{Editor: Hércules Rezende Freitas, Ph.D.}
\date{2024-12-03}

\begin{document}
\maketitle

{
\setcounter{tocdepth}{1}
\tableofcontents
}
\chapter{Sobre o projeto}\label{sobre-o-projeto}

\textbf{O editor}

Licenciado em Ciências, Licenciado em Matemática, Bacharel em Biologia e Bacharel em Nutrição. Especialista em Fitoterapia e em Farmacologia. Possui MBA em Big Data, além de Mestrado e Doutorado em Biofísica pela Universidade Federal do Rio de Janeiro (UFRJ), com grau também pela Universidade de Coimbra. Foi contemplado com bolsas de pesquisa da Coordenação de Aperfeiçoamento de Pessoal de Nível Superior (CAPES), do Conselho Nacional de Desenvolvimento Científico e Tecnológico (CNPq), do National Institutes of Health (NIH) e da Tess Research Foundation (TRF). É autor do livro Princípios de Bioética e coautor dos livros Neurociência da Mente e do Comportamento e Omega Fatty Acids in Brain and Neurological Health. Atua como revisor de periódicos internacionais, como o British Journal of Pharmacology, Frontiers in Neuroscience, Neurochemical Research, BMJ Open, PLOS One e Food Research International. Participa do corpo editorial da revista Neuroenergetics, Nutrition and Brain Health (Frontiers) e é consultor ad hoc de projetos submetidos ao Natural Sciences and Engineering Research Council of Canada (NSERC). Já atuou como professor de nível fundamental e médio na área de Ciências Biológicas no Instituto GayLussac, professor auxiliar em graduação no Centro Universitário IBMR em disciplinas do eixo Ciências Biológicas II (Bioquímica, Biofísica, Farmacologia e Fisiologia) e professor de pós-graduação nas áreas de Fisiologia e Bioquímica no mesmo centro. Foi também conteudista em Fisiologia, Hematologia, Bioquímica e Bioética na UNIGRANRIO e UNIVERSO. Realizou estágio pós-doutoral em Neuropatologia na Universidade da Califórnia (Davis), sob supervisão do Dr.~Lee-Way Jin e da Dra. Izumi Maezawa. Atualmente, é coordenador adjunto na área de saúde e docente DNS I no Centro Universitário IBMR, além de professor auxiliar na Universidade Unigranrio. Possui conhecimento funcional em computação estatística com Python e R.

\emph{Produção científica}

\includegraphics{MCBSbook_files/figure-latex/unnamed-chunk-1-1.pdf}

\textbf{Metodologia científica e Bioestatística}

O projeto ``Metodologia científica e Bioestatística'' é colaborativo e de acesso aberto. O objetivo é que o documento seja sempre atualizado e que seus autores sejam reconhecidos por suas contribuições.

\chapter{Introdução à metodologia científica}\label{introduuxe7uxe3o-uxe0-metodologia-cientuxedfica}

\section{Conceitos básicos de ciência}\label{conceitos-buxe1sicos-de-ciuxeancia}

\section{O método científico}\label{o-muxe9todo-cientuxedfico}

\section{Paradigmas científicos}\label{paradigmas-cientuxedficos}

\section{Ética na pesquisa científica}\label{uxe9tica-na-pesquisa-cientuxedfica}

\chapter{Formulação de problemas e hipóteses}\label{formulauxe7uxe3o-de-problemas-e-hipuxf3teses}

\section{Identificação de problemas de pesquisa}\label{identificauxe7uxe3o-de-problemas-de-pesquisa}

\section{Revisão da literatura}\label{revisuxe3o-da-literatura}

\section{Definição de objetivos}\label{definiuxe7uxe3o-de-objetivos}

\section{Formulação de hipóteses}\label{formulauxe7uxe3o-de-hipuxf3teses}

\chapter{Desenhos de pesquisa}\label{desenhos-de-pesquisa}

\section{Estudos observacionais}\label{estudos-observacionais}

\subsection{Estudos transversais}\label{estudos-transversais}

\subsection{Estudos de coorte}\label{estudos-de-coorte}

\subsection{Estudos de caso-controle}\label{estudos-de-caso-controle}

\section{Ensaios clínicos}\label{ensaios-cluxednicos}

\subsection{Ensaios randomizados controlados}\label{ensaios-randomizados-controlados}

\subsection{Estudos duplo-cegos}\label{estudos-duplo-cegos}

\section{Estudos experimentais}\label{estudos-experimentais}

\subsection{Desenho experimental}\label{desenho-experimental}

\subsection{Controle e aleatorização}\label{controle-e-aleatorizauxe7uxe3o}

\chapter{Bioestatística descritiva}\label{bioestatuxedstica-descritiva}

\section{Medidas de tendência central}\label{medidas-de-tenduxeancia-central}

\subsection{Média}\label{muxe9dia}

\subsection{Mediana}\label{mediana}

\subsection{Moda}\label{moda}

\section{Medidas de dispersão}\label{medidas-de-dispersuxe3o}

\subsection{Variância}\label{variuxe2ncia}

\subsection{Desvio padrão}\label{desvio-padruxe3o}

\subsection{Amplitude}\label{amplitude}

\section{Distribuições de frequência}\label{distribuiuxe7uxf5es-de-frequuxeancia}

\section{Representações gráficas}\label{representauxe7uxf5es-gruxe1ficas}

\subsection{Histogramas}\label{histogramas}

\subsection{Boxplots}\label{boxplots}

\subsection{Gráficos de dispersão}\label{gruxe1ficos-de-dispersuxe3o}

\chapter{Bioestatística inferencial}\label{bioestatuxedstica-inferencial}

\section{Conceitos de probabilidade}\label{conceitos-de-probabilidade}

\section{Distribuições de probabilidade}\label{distribuiuxe7uxf5es-de-probabilidade}

\subsection{Distribuição normal}\label{distribuiuxe7uxe3o-normal}

\subsection{Distribuição t de student}\label{distribuiuxe7uxe3o-t-de-student}

\subsection{Distribuição qui-quadrado}\label{distribuiuxe7uxe3o-qui-quadrado}

\section{Estimação e intervalos de confiança}\label{estimauxe7uxe3o-e-intervalos-de-confianuxe7a}

\section{Testes de hipóteses}\label{testes-de-hipuxf3teses}

\subsection{Testes paramétricos}\label{testes-paramuxe9tricos}

\subsection{Testes não-paramétricos}\label{testes-nuxe3o-paramuxe9tricos}

\chapter{Análise de dados}\label{anuxe1lise-de-dados}

\section{Correlação e regressão}\label{correlauxe7uxe3o-e-regressuxe3o}

\subsection{Correlação de Pearson e Spearman}\label{correlauxe7uxe3o-de-pearson-e-spearman}

\subsection{Regressão linear simples e múltipla}\label{regressuxe3o-linear-simples-e-muxfaltipla}

\section{Análise de variância (ANOVA)}\label{anuxe1lise-de-variuxe2ncia-anova}

\subsection{ANOVA de uma via}\label{anova-de-uma-via}

\subsection{ANOVA fatorial}\label{anova-fatorial}

\section{Análise multivariada}\label{anuxe1lise-multivariada}

\subsection{Componentes principais}\label{componentes-principais}

\subsection{Clusterização}\label{clusterizauxe7uxe3o}

\chapter{\texorpdfstring{Uso de \emph{softwares} estatísticos}{Uso de softwares estatísticos}}\label{uso-de-softwares-estatuxedsticos}

\section{\texorpdfstring{\emph{Softwares} ``no-code'' e ``low-code''}{Softwares ``no-code'' e ``low-code''}}\label{softwares-no-code-e-low-code}

\section{R para análise estatística}\label{r-para-anuxe1lise-estatuxedstica}

Dr.~Hércules Freitas, Ph.D.\(^1\)

\(^1\)Universidade Federal do Rio de Janeiro (UFRJ), Centro de Ciências da Saúde, Instituto de Bioquímica Leopoldo de Meis, Laboratório de Erros Inatos do Metabolismo, Rio de Janeiro/RJ, Brasil.

\subsection{Aprendendo a programar}\label{aprendendo-a-programar}

Aprender a programar, especialmente em uma linguagem como R, destinada à ciência de dados, é uma jornada empolgante, mas também repleta de desafios. Um dos maiores obstáculos que os iniciantes enfrentam não é a complexidade dos conceitos ou a sintaxe da linguagem, mas a frustração que surge ao encontrar erros e \emph{bugs}. Este sentimento, embora desagradável, é uma parte integral do processo de aprendizado. A chave para superar essa frustração não está em evitá-la, mas em aprender a lidar com ela de maneira produtiva.

\subsubsection{A importância de lidar com a frustração}\label{a-importuxe2ncia-de-lidar-com-a-frustrauxe7uxe3o}

A frustração, embora desconfortável, é um sinal de que você está se desafiando e saindo da sua zona de conforto. É um indicativo de crescimento e aprendizado. Quando você se depara com um erro em seu código R, é fácil sentir-se desanimado ou questionar suas habilidades. No entanto, é crucial reconhecer que até os programadores mais experientes enfrentam erros e \emph{bugs} regularmente. A diferença está em como eles reagem a esses contratempos.

Uma habilidade subestimada na programação é a capacidade de ler e interpretar mensagens de erro. Em R, como em muitas outras linguagens de programação, as mensagens de erro são projetadas para guiar o programador na identificação e correção de problemas. Embora possam parecer crípticas no início, com a prática, você começará a reconhecer padrões comuns e entender melhor o que essas mensagens estão tentando comunicar. Abaixo, seguem algumas dicas para lidar com as dificuldades no aprendizado inicial:

\begin{enumerate}
\def\labelenumi{\arabic{enumi}.}
\item
  \textbf{Aceitação}: aceite que a frustração é uma parte normal do processo de aprendizado. Reconheça seus sentimentos, mas não permita que eles dominem sua motivação para aprender;
\item
  \textbf{Pausas estratégicas}: quando se sentir sobrecarregado, faça uma pausa. Distanciar-se temporariamente do problema pode ajudar a clarear sua mente e trazer novas perspectivas;
\item
  \textbf{Comunidade e suporte}: não hesite em buscar ajuda. A comunidade R é vasta e acolhedora, com fóruns e grupos dedicados a ajudar programadores de todos os níveis. Compartilhar suas dificuldades e buscar conselhos pode ser incrivelmente útil;
\item
  \textbf{Prática e persistência}: a prática contínua é fundamental. Quanto mais você se expõe a diferentes problemas e erros, mais equipado estará para lidar com eles no futuro;
\item
  \textbf{Celebre pequenas vitórias}: cada erro corrigido é um passo à frente em sua jornada de aprendizado. Celebre essas conquistas e reconheça seu progresso.
\end{enumerate}

Aprender a programar em R é uma jornada valiosa que abre portas para o mundo da análise de dados. Embora a frustração seja uma parte inevitável desse processo, adotar uma abordagem positiva e resiliente pode transformar esses desafios em oportunidades de aprendizado. Lembre-se, cada erro é uma chance de crescer, e cada problema resolvido é um passo em direção à maestria na programação. Mantenha-se motivado, pratique regularmente e, mais importante, seja paciente consigo mesmo. O caminho para se tornar proficiente em R pode ser longo, mas é, sem dúvida, recompensador.

\subsection{Visão geral do R e suas aplicações}\label{visuxe3o-geral-do-r-e-suas-aplicauxe7uxf5es}

R é uma linguagem de programação e um ambiente de software para análise estatística e gráfica. A história do R remonta à década de 1970 com o desenvolvimento da linguagem S no Bell Labs por John Chambers e outros (\citeproc{ref-chambers1980}{Chambers 1980}). A linguagem S foi projetada para ser uma linguagem de programação que fosse tanto eficiente quanto fácil de usar, com foco em análise de dados e gráficos estatísticos.

Na década de 1990, Ross Ihaka e Robert Gentleman, da Universidade de Auckland, na Nova Zelândia, iniciaram o desenvolvimento do R como um projeto de pesquisa. Eles foram influenciados pela linguagem S e pelo Scheme, uma linguagem de programação com semântica de escopo léxico. O R foi concebido como um dialeto da linguagem S, com a intenção de melhorar e ampliar a análise estatística e as capacidades gráficas (\citeproc{ref-ihaka1996}{Ihaka and Gentleman 1996}).

R é uma linguagem de programação interpretada, o que significa que o código é executado diretamente, sem a necessidade de compilação prévia. Isso facilita a depuração e o desenvolvimento iterativo. Além disso, R é uma linguagem dinamicamente tipada, permitindo que os tipos de dados sejam alterados em tempo de execução.

Uma das principais características do R é a sua extensibilidade. A comunidade de usuários e desenvolvedores contribuiu com uma vasta coleção de pacotes que estendem a funcionalidade básica do R, disponíveis através do \emph{Comprehensive R Archive Network (CRAN)} (\citeproc{ref-hornik2012}{Hornik 2012}). Esses pacotes cobrem uma ampla gama de técnicas estatísticas, modelos gráficos, métodos de \emph{machine learning}, e muito mais.

R também é conhecido por sua capacidade de produzir gráficos de alta qualidade, que são essenciais para a análise exploratória de dados e a apresentação de resultados estatísticos. A linguagem oferece diversas funções e pacotes para a criação de gráficos, incluindo o popular \emph{ggplot2}, que permite a construção de gráficos complexos de maneira intuitiva (\citeproc{ref-wickham2011}{Wickham 2011}).

Outra característica importante do R é a sua comunidade ativa e colaborativa. Os usuários de R frequentemente compartilham seu código e experiências, ajudando uns aos outros a resolver problemas e a melhorar suas habilidades de programação. Isso é facilitado por fóruns de discussão, listas de e-mail e conferências.

As principais aplicações da linguagem R no mundo do trabalho abrangem uma vasta gama de setores, especialmente aqueles que dependem intensamente da análise de dados, estatística e modelagem preditiva. Profissões como cientistas de dados, analistas de mercado, pesquisadores em saúde e biologia, e especialistas em \emph{machine learning} são apenas alguns exemplos de carreiras que se beneficiam diretamente do uso de R. Esta linguagem é particularmente valorizada por sua capacidade de manipular grandes conjuntos de dados, realizar análises estatísticas complexas, criar visualizações de dados avançadas e desenvolver modelos de \emph{machine learning}.

No futuro, espera-se que a demanda por profissionais com habilidades em R continue a crescer, à medida que mais setores reconhecem a importância da tomada de decisões baseada em dados. Profissões emergentes, como especialistas em big data, analistas de cibersegurança que utilizam técnicas de aprendizado de máquina para identificar ameaças, e profissionais envolvidos na criação e gestão de ambientes virtuais no metaverso, também poderão se beneficiar do uso de R.

Além disso, à medida que a inteligência artificial (IA) e a análise de dados se tornam cada vez mais integradas em diferentes aspectos do mundo do trabalho, a capacidade de utilizar R para desenvolver e implementar soluções baseadas em dados será uma habilidade altamente valorizada.

A linguagem R, com sua comunidade ativa e vasta coleção de pacotes, oferece uma plataforma robusta para inovação e desenvolvimento em diversas áreas. Profissionais do futuro poderão utilizar R para explorar novos territórios em ciência de dados, otimização de processos, desenvolvimento de produtos baseados em IA, análises ambientais e muito mais. A flexibilidade e a capacidade de adaptação de R a diferentes contextos e desafios tornam-na uma ferramenta valiosa para profissionais que buscam estar na vanguarda da inovação e da análise de dados.

\subsection{\texorpdfstring{\textbf{Instalando o R e o RStudio}~}{Instalando o R e o RStudio~}}\label{instalando-o-r-e-o-rstudio}

Para instalar o R e o RStudio em computadores com sistemas operacionais Windows ou Mac, siga os passos abaixo:~~

\subsubsection{Instalação do R no Windows:~}\label{instalauxe7uxe3o-do-r-no-windows}

\begin{enumerate}
\def\labelenumi{\arabic{enumi}.}
\item
  Acesse o site do CRAN (\href{https://cran.r-project.org}{Comprehensive R Archive Network}), que é uma rede de servidores que armazena versões atualizadas do R;~
\item
  Em ``\emph{Download and Install R}'', clique em ``\emph{Download R for Windows}'';~
\item
  Clique em ``\emph{\ldots install R for the first time}'';
\item
  Clique em ``\emph{Download R x.x.x for Windows}'', onde ``x.x.x'' é o número da versão mais recente;

  \includegraphics{images/clipboard-1429771558.png}
\item
  Após o \emph{download}, execute o arquivo baixado e siga as instruções do instalador. Durante a instalação, você pode aceitar as configurações padrão, que são adequadas para a maioria dos usuários.
\end{enumerate}

\subsubsection{Instalação do RStudio no Windows}\label{instalauxe7uxe3o-do-rstudio-no-windows}

\begin{enumerate}
\def\labelenumi{\arabic{enumi}.}
\item
  Após instalar o R, acesse o site do \href{https://posit.co/download/rstudio-desktop/}{RStudio (Posit)};
\item
  Desça a página e clique em ``\emph{Download RStudio Desktop for Windows}'';

  \includegraphics[width=3.47917in,height=\textheight]{images/clipboard-3306259698.png}
\item
  Execute o arquivo baixado e siga as instruções para concluir a instalação.
\end{enumerate}

\subsubsection{Instalação do R no Mac}\label{instalauxe7uxe3o-do-r-no-mac}

\begin{enumerate}
\def\labelenumi{\arabic{enumi}.}
\item
  Acesse o site do \href{https://cran.r-project.org}{CRAN};
\item
  Clique em ``\emph{Download R for macOS}'';

  \includegraphics{images/clipboard-2368293499.png}
\item
  Escolha a versão do R que deseja instalar (\emph{Macbooks Intell, mais antigos, ou Macbooks M1/M2});

  \includegraphics[width=2.69792in,height=\textheight]{images/clipboard-3553229090.png}
\item
  Após o \emph{download}, abra o arquivo ``.pkg'' baixado e siga as instruções para instalar o R.
\end{enumerate}

\subsubsection{Instalação do RStudio no Mac}\label{instalauxe7uxe3o-do-rstudio-no-mac}

\begin{enumerate}
\def\labelenumi{\arabic{enumi}.}
\item
  Visite o site do \href{https://posit.co/download/rstudio-desktop/}{RStudio (Posit)} após instalar o R;
\item
  Clique em ``\emph{Download RStudio}'';
\item
  Na seção \emph{RStudio Desktop}, clique em ``\emph{Download RStudio Desktop for Mac}'';
\item
  Escolha a versão apropriada para Mac, seja para processadores Intel ou M1, e inicie o \emph{download};
\item
  Abra o arquivo baixado e siga as instruções para instalar o RStudio.
\end{enumerate}

Após a instalação, você estará pronto(a) para utilizar todas as funcionalidades do R no RStudio.

\includegraphics[width=5.25in,height=\textheight]{images/clipboard-4172689374.png}

\subsection{Sintaxe básica e operações no R}\label{sintaxe-buxe1sica-e-operauxe7uxf5es-no-r}

\subsubsection{Sintaxe básica}\label{sintaxe-buxe1sica}

A sintaxe do R é bastante simples e direta. A linguagem faz distinção entre maiúsculas e minúsculas, portanto, ``A'' e ``a'' são considerados símbolos diferentes e podem se referir a variáveis diferentes. Os comandos no R são expressões ou atribuições. Se um comando é uma expressão, seu valor é calculado e visualizado, mas é perdido em seguida. Uma atribuição, por outro lado, calcula a expressão e atribui o resultado a uma variável, que é salva no ambiente de trabalho do R. Aqui está um exemplo de atribuição a variável no R (``x \textless- 1'' ou ``x é igul a 1''):

\textbf{Passo 1}: digite o código desejado no editor

\includegraphics{images/clipboard-474586210.png}

* O código acima indica, na primeira linha, ``X contém o valor 1''. A segunda linha ``pede'' ao programa que indique o valor de ``X''.

\textbf{Passo 2}: selecione o código a ser executado e aperte ``Ctrl + Enter''

\includegraphics{images/clipboard-2485386777.png}

*O resultado é apresentado no Console (i.e., ``{[}1{]} 1'' ou ``X contém um valor, e esse valor é 1'').

\textbf{Passo 3}: use o objeto X gerado em outros códigos

\includegraphics{images/clipboard-3669577123.png}

*''X'' é um objeto de valor 1. Se somado a 2, retorna o valor 3.

Agora, tente interagir com os exemplos abaixo:

\begin{enumerate}
\def\labelenumi{\arabic{enumi}.}
\item
  \textbf{Criação de vetores}: v \textless- c(1, 2, 3, 4, 5)
\item
  \textbf{Criação de matrizes}: m \textless- matrix(1:9, nrow = 3, ncol = 3)
\item
  \textbf{Criação de listas}: l \textless- list(nome = ``João'', idade = 30, altura = 1.75)
\item
  \textbf{Criação de data frames}: df \textless- data.frame(nome = c(``João'', ``Maria''), idade = c(30, 25))
\end{enumerate}

\textbf{Exemplo}: produzindo um \emph{data frame} (quadro de dados)

\includegraphics{images/clipboard-3472985925.png}

\subsubsection{Operações básicas}\label{operauxe7uxf5es-buxe1sicas}

O R pode ser usado como uma calculadora simples, realizando operações aritméticas básicas como adição (+), subtração (-), multiplicação (*), divisão (/) e potenciação (\^{}). Além disso, o R também suporta operadores relacionais como menor (\textless), menor ou igual (\textless=), maior (\textgreater), maior ou igual (\textgreater=), igual (==) e diferente (!=).

Aqui estão alguns exemplos de operações básicas no R:

\begin{enumerate}
\def\labelenumi{\arabic{enumi}.}
\item
  \textbf{Adição}: ``3 + 2'' resulta em 5
\item
  \textbf{Subtração}: ``5 - 2'' resulta em 3
\item
  \textbf{Multiplicação}: ``3 * 2'' resulta em 6
\item
  \textbf{Divisão}: ``6 / 2'' resulta em 3
\item
  \textbf{Potenciação}: ``2 \^{} 3'' resulta em 8
\end{enumerate}

Além disso, o R suporta operações com vetores e matrizes. Por exemplo, se você tem dois vetores de mesmo comprimento, pode somá-los diretamente: ``c(1, 2, 3) + c(4, 5, 6)'' resulta em ``c(5, 7, 9)''.

\subsubsection{Manipulação de dados}\label{manipulauxe7uxe3o-de-dados}

O R também oferece uma variedade de funções para manipulação de dados. Por exemplo, você pode acessar elementos de um vetor usando o operador de colchetes ({[}{]}). Se você tem um vetor ``x \textless- c(1, 2, 3, 4, 5)'', pode acessar o terceiro elemento com ``x{[}3{]}'', que resulta em 3.

Além disso, o R permite a manipulação de \emph{strings}. Por exemplo, você pode criar duas variáveis que armazenam a primeira letra do seu primeiro e segundo nome, e então compará-las usando operadores lógicos.

Esses são apenas alguns exemplos da sintaxe básica e operações no R. A linguagem R é extremamente poderosa e flexível, permitindo uma ampla gama de manipulações de dados e análises estatísticas. Não se preocupe se encontrou erros ou se ainda não conseguiu utilizar todos os códigos de exemplo. Continue lendo o material do curso e praticando, é o melhor caminho para o aprendizado eficaz da linguagem R.

\subsection{Escrevendo código eficiente e legível}\label{escrevendo-cuxf3digo-eficiente-e-leguxedvel}

Escrever código eficiente e legível em R é fundamental para garantir a manutenção, compreensão e colaboração em projetos de programação. A linguagem R, amplamente utilizada em estatística e análise de dados, oferece diversas funcionalidades que, quando bem aproveitadas, podem melhorar significativamente a qualidade do código. Abaixo, apresentamos algumas recomendações práticas acompanhadas de exemplos para escrever um código mais limpo, eficiente e legível em R.

\subsubsection{Convenções de nomenclatura}\label{convenuxe7uxf5es-de-nomenclatura}

Utilize nomes significativos e autoexplicativos para variáveis e funções. Isso facilita a compreensão do propósito de cada componente do código:

\includegraphics{images/clipboard-2315088477.png}

\subsubsection{Formatação consistente}\label{formatauxe7uxe3o-consistente}

Mantenha um padrão de recuo, espaçamento e formatação. Isso torna o código mais organizado e fácil de ler:

\includegraphics{images/clipboard-2461203963.png}

\subsubsection{Comentários e documentação}\label{comentuxe1rios-e-documentauxe7uxe3o}

Inclua comentários relevantes que expliquem o propósito e a funcionalidade do código. Evite comentários óbvios e mantenha-os atualizados:

\includegraphics{images/clipboard-978422987.png}

\subsubsection{Simplicidade e modularidade}\label{simplicidade-e-modularidade}

Evite complexidade desnecessária. Divida o código em funções e módulos pequenos e reutilizáveis:

\includegraphics{images/clipboard-1708113605.png}

\subsubsection{Tratamento de erros}\label{tratamento-de-erros}

Manipule e documente adequadamente quaisquer condições de erro para evitar falhas inesperadas:

\includegraphics{images/clipboard-2761045831.png}

\subsubsection{Evitar duplicações de código}\label{evitar-duplicauxe7uxf5es-de-cuxf3digo}

Reutilize trechos de código e crie funções ou classes para evitar repetições desnecessárias:

\includegraphics{images/clipboard-491234479.png}

\subsubsection{Vetorização}\label{vetorizauxe7uxe3o}

R é uma linguagem vetorizada, o que significa que muitas operações podem ser realizadas sem o uso explícito de \emph{loops}, tornando o código mais eficiente:

\includegraphics{images/clipboard-2187533748.png}

Seguindo essas práticas, é possível escrever código em R que não apenas atenda aos requisitos funcionais, mas também seja fácil de ler, entender e manter. A chave é sempre buscar a clareza, a simplicidade e a eficiência, facilitando assim o trabalho tanto do autor do código quanto de outros desenvolvedores que possam trabalhar com ele no futuro.

\subsection{Depuração e tratamento de erros em R}\label{depurauxe7uxe3o-e-tratamento-de-erros-em-r}

Escrever código que funcione corretamente é apenas uma parte do desenvolvimento de software; a outra parte é garantir que o código continue funcionando e que possa ser corrigido quando surgirem problemas. A depuração e o tratamento de erros são habilidades essenciais para qualquer programador, e na linguagem R não é diferente. Abaixo estão algumas estratégias e ferramentas para depuração e tratamento de erros em R, acompanhadas de exemplos práticos.

\subsubsection{Depuração interativa}\label{depurauxe7uxe3o-interativa}

A depuração interativa permite que você inspecione o estado do seu programa R enquanto ele está sendo executado. O RStudio oferece ferramentas integradas para isso, como o inspetor de erros e a função `traceback( ) `, que lista a sequência de chamadas que levaram ao erro. Além disso, o RStudio possui ferramentas como ``Rerun with Debug'' e `options(error = browser())`, que abrem uma sessão interativa onde o erro ocorreu.

\includegraphics{images/clipboard-3416304030.png}

\subsubsection{Tratamento de erros com tryCatch}\label{tratamento-de-erros-com-trycatch}

A função `tryCatch()` é a principal ferramenta para lidar com erros e avisos em R. Ela permite que você especifique funções manipuladoras que controlam o que acontece quando uma condição é sinalizada.

\includegraphics{images/clipboard-3819030973.png}

\subsubsection{Logging de erros}\label{logging-de-erros}

Registrar erros é uma prática padrão no desenvolvimento de software. Em R, você pode usar pacotes de \emph{logging} ou simplesmente registrar erros em um arquivo JSON ou em um banco de dados. Isso é especialmente útil em aplicativos de produção, como aplicativos shiny ou APIs REST.

\includegraphics{images/clipboard-2118828263.png}

\subsubsection{Debugging com print}\label{debugging-com-print}

Se as ferramentas de depuração não ajudarem, uma boa alternativa é o ``print debugging'', onde você insere vários comandos `print()` para localizar precisamente o problema e ver os valores das variáveis importantes. É um método lento e primitivo, mas sempre funciona.

\includegraphics{images/clipboard-2905347804.png}

A depuração e o tratamento de erros são partes cruciais do desenvolvimento em R. Utilizar as ferramentas e estratégias corretas pode economizar tempo e evitar a perda de dados ou resultados importantes. É essencial criar mensagens de erro claras e informativas e documentar o significado dos erros que seu software pode gerar. A experiência e a prática levarão a uma maior maestria na depuração e no tratamento de erros em R.

\subsection{Documentando e compartilhando seu código}\label{documentando-e-compartilhando-seu-cuxf3digo}

Documentar e compartilhar seu código são práticas essenciais para qualquer programador, especialmente para iniciantes em programação que estão aprendendo a linguagem R. Essas práticas não apenas facilitam a colaboração e a revisão por outros, mas também ajudam o próprio autor a entender e manter seu código ao longo do tempo.

\subsubsection{Documentação de código}\label{documentauxe7uxe3o-de-cuxf3digo}

A documentação é a primeira linha de comunicação entre o programador e quem vai ler o código depois, seja o próprio autor em um momento futuro ou outros programadores. Em R, a documentação pode ser feita diretamente no código, usando comentários, e de forma mais estruturada, usando pacotes como o `roxygen2` (\citeproc{ref-wickham2024}{Wickham et al. 2024}), que permite criar uma documentação que pode ser convertida em um manual de ajuda para o pacote.

\includegraphics{images/clipboard-3987177554.png}

\subsubsection{Compartilhamento de código}\label{compartilhamento-de-cuxf3digo}

O compartilhamento de código em R pode ser feito de várias maneiras. Uma das mais comuns é através de scripts, que são arquivos de texto com extensão `.R`. Esses scripts podem ser compartilhados por e-mail, repositórios de código como GitHub ou plataformas como \href{https://rpubs.com}{RPubs}, onde é possível publicar análises e códigos para que outros possam ver e usar.

\subsubsection{R Markdown}\label{r-markdown}

Uma ferramenta poderosa para documentar e compartilhar análises em R é o R Markdown (\citeproc{ref-baumer2015}{Baumer and Udwin 2015}). Com ele, é possível combinar texto narrativo com código R que pode ser executado para gerar resultados, que são automaticamente incluídos no documento. Isso é particularmente útil para relatórios, onde a análise e os resultados precisam ser apresentados de forma clara e reproduzível.

\includegraphics{images/clipboard-1117062658.png}

\subsection{Introdução à análise estatística com R}\label{introduuxe7uxe3o-uxe0-anuxe1lise-estatuxedstica-com-r}

A análise estatística é uma parte fundamental da linguagem de programação R, que foi originalmente desenvolvida com um forte foco em estatística e análise de dados. Aqui está uma introdução detalhada à análise estatística com R.

R é uma linguagem de programação e um ambiente de software para análise estatística e gráficos. Ele fornece uma ampla variedade de técnicas estatísticas, incluindo regressão linear e não linear, análise de séries temporais, classificação, agrupamento e muito mais. Além disso, R é altamente extensível através de pacotes, que são bibliotecas de funções desenvolvidas pela comunidade para estender a funcionalidade do R.

\subsubsection{Estatística Descritiva}\label{estatuxedstica-descritiva}

A estatística descritiva é a primeira etapa na análise de dados. Ela envolve resumir e organizar os dados de maneira que possam ser facilmente compreendidos. As funções básicas do R para estatística descritiva incluem `mean()` para calcular a média, `median()` para a mediana, `sd()` para o desvio padrão, `var()` para a variância, `min()` e `max()` para os valores mínimo e máximo, respectivamente, e `summary()` para obter um resumo estatístico dos dados. As funções mencionadas acima já foram exemplificadas no módulo 2. Tente aplicar essas funções em um novo exemplo (ex.: vet1 \textless- c(1,4,8,3,4,6,7,8,2,6,8,9,2,1,)).

\subsubsection{Visualização de dados}\label{visualizauxe7uxe3o-de-dados}

A visualização de dados é uma parte importante da análise estatística. R fornece várias ferramentas para criar gráficos e visualizações de dados. As funções básicas incluem `plot()` para criar gráficos de dispersão, `hist()` para histogramas, `boxplot()` para boxplots e `barplot()` para gráficos de barras. Além disso, o pacote `ggplot2` oferece uma poderosa e flexível estrutura para criar gráficos complexos. O pacote `ggplot2` será explorado em mais detalhes no próximo módulo.

\textbf{Exemplo}: usando a função ``plot( )''

\includegraphics{images/clipboard-3607661657.png}

*Dois vetores, ``Hora'' e ``Temperatura'', foram criados e depois inseridos na função ``plot( )''. Note que a sintaxe da função foi ``plot(Temperatura \textasciitilde{} Hora)'', que é o equivalente de dizer ``Temperatura em função da Hora'' em linguagem R. O mesmo resultado pode ser obtido informando qual vetor ocupa qual eixo: ``plot(x = Hora, y = Temperatura)''.

\textbf{Exemplo}: criando um histograma com a função ``hist( )''

\includegraphics{images/clipboard-680965483.png}

*Utilizando o vetor ``Temperatura'' do exemplo anterior, criou-se um histograma com a função ``hist( )''. Note como temperaturas entre 23,0 e 23,5 ºC são mais comuns do que as outras medidas ao longo do intervalo de tempo avaliado.

\textbf{Exemplo}: criando um boxplot com a função ``boxplot( )''

\includegraphics{images/clipboard-1951655399.png}

*Utilizando a mesma estratégia do exemplo anterior, um diagrama de caixa (\emph{boxplot}) foi criado através da função ``boxplot( )''. Se desejar, leia mais sobre \href{https://statplace.com.br/blog/como-interpretar-um-boxplot/}{como interpretar um \emph{boxplot}}, e pratique com novos exercícios.

\subsubsection{Estatística inferencial}\label{estatuxedstica-inferencial}

A estatística inferencial é o processo de usar os dados de uma amostra para inferir propriedades da população. Isso envolve a realização de testes de hipóteses para determinar se um resultado observado é devido ao acaso ou a algum fator subjacente. As funções básicas do R para estatística inferencial incluem `t.test()` para o teste t, `chisq.test()` para o teste do qui-quadrado, `cor.test()` para o teste de correlação, e `lm()` para a regressão linear.

\textbf{Exemplo}: realizando um teste de hipóteses com dois vetores numéricos

\includegraphics{images/clipboard-2899251105.png}

*Suponhamos que as temperaturas superficiais de duas lâmpadas tenham sido medidas repetidas vezes. Agora, podemos comparar esses vetores numéricos (``T\_lampada1'' e ``T\_lampada2'') considerando a hipótese de que a diferença verdadeira entre as médias de ambos os vetores é igual a 0. Como se observa, a função ``t.test( )'' retorna uma probabilidade muito baixa (\emph{p} = 0.00001152) de que os dois vetores tenham surgido de uma mesma distribuição. Em outras palavras, podemos atribuir confiança razoável de que ambas as lâmpadas possuam temperaturas médias distintas.

\subsubsection{Modelagem estatística}\label{modelagem-estatuxedstica}

A modelagem estatística é o processo de desenvolver um modelo matemático que descreve uma relação entre variáveis. R fornece várias funções para modelagem estatística, incluindo `lm()` para regressão linear, `glm()` para regressão linear generalizada, `anova()` para análise de variância, e `arima()` para modelagem de séries temporais.

\textbf{Exemplo}: avaliando a correlação entre dois vetores numéricos com a função ``lm( )''

\includegraphics{images/clipboard-870011745.png}

*Dados de estatura e massa corporal de diversas pessoas foram armazenados em vetores numéricos. Um gráfico da relação entre as duas variáveis foi gerado com o código ``plot(Massa\_corporal \textasciitilde{} Estatura)'', seguido da linha vermelha ``abline(line(Massa\_corporal \textasciitilde{} Estatura), col=''red'')''. Como a relação entre estatura e massa corporal parecem proporcionais, a função ``lm( )'' foi utilizada para medir a associação entre os dois vetores. De acordo com o valor de R-quadrado (``\emph{R-squared}'') da função ``summary(lm(Massa\_corporal \textasciitilde{} Estatura))'', 97 \% da variação na massa corporal pode ser explicada pela variação observada na estatura. No exemplo, a função ``summary( )'' é usada para detalhar os resultados do modelo de regressão linear calculado pela função ``lm( )''.

R é uma ferramenta poderosa para análise estatística, oferecendo uma ampla gama de funções para estatística descritiva, visualização de dados, estatística inferencial e modelagem estatística. Aprender a usar R para análise estatística pode abrir novas oportunidades para a exploração e compreensão de dados.

\subsection{Usando R para estatísticas descritivas}\label{usando-r-para-estatuxedsticas-descritivas}

A linguagem de programação R é uma ferramenta poderosa para a realização de estatísticas descritivas, que são métodos utilizados para resumir e organizar um conjunto de dados de maneira que possam ser facilmente compreendidos. Abaixo, você encontrará uma explanação detalhada de como usar R para estatísticas descritivas. Exemplos dessas funções podem ser encontradas no módulo 2, mas é recomendado que você pratique os códigos deste módulo em seu ambiente RStudio.

\subsubsection{Medidas de tendência central}\label{medidas-de-tenduxeancia-central-1}

As medidas de tendência central são estatísticas que indicam onde os dados estão centrados. As principais medidas de tendência central são a média, a mediana e a moda.

\begin{enumerate}
\def\labelenumi{\arabic{enumi}.}
\item
  \textbf{Média}: a média de um conjunto de dados é calculada somando todos os valores e dividindo pelo número de valores. No R, a média é calculada usando a função `mean()`. Por exemplo, `mean(c(1, 2, 3, 4, 5))` calcula a média do vetor de números de 1 a 5.
\item
  \textbf{Mediana}: a mediana é o valor que divide os dados ao meio quando estão ordenados. No R, a mediana é calculada usando a função `median()`. Por exemplo, `median(c(1, 2, 3, 4, 5))` calcula a mediana do vetor de números de 1 a 5.
\item
  \textbf{Moda}: a moda é o valor mais frequente em um conjunto de dados. R não tem uma função embutida para calcular a moda, mas pode ser calculada usando funções de outros pacotes ou escrevendo sua própria função.
\end{enumerate}

\subsubsection{Medidas de dispersão}\label{medidas-de-dispersuxe3o-1}

As medidas de dispersão são estatísticas que indicam o quanto os dados estão espalhados. As principais medidas de dispersão são a variância, o desvio padrão e a amplitude.

\begin{enumerate}
\def\labelenumi{\arabic{enumi}.}
\item
  \textbf{Variância}: a variância é uma medida da dispersão que indica o quanto os valores se desviam da média. No R, a variância é calculada usando a função `var()`. Por exemplo, `var(c(1, 2, 3, 4, 5))` calcula a variância do vetor de números de 1 a 5.
\item
  \textbf{Desvio padrão}: o desvio padrão é a raiz quadrada da variância e fornece uma medida de dispersão que está na mesma unidade que os dados. No R, o desvio padrão é calculado usando a função `sd()`. Por exemplo, `sd(c(1, 2, 3, 4, 5))` calcula o desvio padrão do vetor de números de 1 a 5.
\item
  \textbf{Amplitude}: a amplitude é a diferença entre o maior e o menor valor em um conjunto de dados. No R, a amplitude pode ser calculada subtraindo o resultado da função `min()` do resultado da função `max()`. Por exemplo, `max(c(1, 2, 3, 4, 5)) - min(c(1, 2, 3, 4, 5))` calcula a amplitude do vetor de números de 1 a 5.
\end{enumerate}

\subsubsection{Resumo dos dados}\label{resumo-dos-dados}

A função `summary()` no R fornece um resumo estatístico dos dados, incluindo a média, a mediana, o mínimo, o máximo, o primeiro quartil e o terceiro quartil. Por exemplo, `summary(c(1, 2, 3, 4, 5))` fornece um resumo do vetor de números de 1 a 5.

R é uma ferramenta poderosa para estatísticas descritivas, oferecendo uma ampla gama de funções para calcular medidas de tendência central, medidas de dispersão e resumos de dados. Aprender a usar R para estatísticas descritivas pode abrir novas oportunidades para a exploração e compreensão de dados.

\subsection{Estatísticas inferenciais com R}\label{estatuxedsticas-inferenciais-com-r}

As estatísticas inferenciais com R envolvem o uso de técnicas estatísticas para fazer generalizações sobre uma população com base em uma amostra de dados. Essas técnicas são fundamentais para a tomada de decisões baseada em dados e para a pesquisa científica.

\subsubsection{Conceitos básicos}\label{conceitos-buxe1sicos}

Antes de realizar a inferência estatística, é importante entender alguns conceitos básicos, como população, amostra, parâmetro e estimativa. A população é o conjunto completo de observações que estão sendo estudadas, enquanto uma amostra é um subconjunto dessa população. Um parâmetro é uma medida resumida da população, como a média populacional, e uma estimativa é o correspondente calculado a partir da amostra.

\subsubsection{Testes de hipóteses}\label{testes-de-hipuxf3teses-1}

Um dos pilares da inferência estatística é o teste de hipóteses. O objetivo é testar uma afirmação (hipótese) sobre um parâmetro populacional. No R, funções como `t.test()`, `chisq.test()` e `anova()` são usadas para realizar diferentes tipos de testes de hipóteses.

Por exemplo, o `t.test()` é usado para comparar as médias de duas amostras e determinar se elas são significativamente diferentes. O teste do qui-quadrado (`chisq.test()`) é frequentemente usado para testar a independência entre duas variáveis categóricas. A análise de variância (ANOVA), realizada através da função `anova()`, é usada para comparar as médias de três ou mais grupos.

\subsubsection{Intervalos de confiança}\label{intervalos-de-confianuxe7a}

Outra ferramenta importante na inferência estatística é o intervalo de confiança, que fornece um intervalo estimado dentro do qual é provável que o parâmetro populacional esteja. No R, o `t.test()` e outras funções de teste fornecem intervalos de confiança como parte de seus resultados.

\subsubsection{Regressão e correlação}\label{regressuxe3o-e-correlauxe7uxe3o}

A análise de regressão é usada para entender a relação entre variáveis. No R, a função `lm()` é usada para realizar regressão linear, que modela a relação entre uma variável dependente e uma ou mais variáveis independentes. A correlação, que mede a força e a direção da relação linear entre duas variáveis, pode ser calculada com a função `cor()`.

\subsubsection{Análise de séries temporais}\label{anuxe1lise-de-suxe9ries-temporais}

A análise de séries temporais é um tipo de inferência estatística que lida com dados coletados ao longo do tempo. No R, funções como `ts()` para criar objetos de séries temporais e `arima()` para modelagem de séries temporais são usadas para analisar como os dados mudam ao longo do tempo.

\textbf{Exemplo}: utilizando a função ``arima( )'' para prever uma série temporal

\includegraphics{images/clipboard-3461101962.png}

*Utilizando os dados de tempo e variação do preço de uma determinada ação, podemos aplicar a função ``arima( )'' para criar um modelo e prever a variação de preço para as próximas 3 horas. Veja como o gráfico representa a série temporal original seguida de três pontos (azuis) informando a variação futura prevista.

A inferência estatística no R é uma área vasta e complexa, que requer um entendimento sólido de teoria estatística e a habilidade de aplicar essa teoria usando o R. As funções e pacotes disponíveis no R tornam a linguagem uma ferramenta poderosa para realizar análises inferenciais e extrair insights significativos de dados amostrais.

\subsection{Fundamentos da visualização de dados em R}\label{fundamentos-da-visualizauxe7uxe3o-de-dados-em-r}

A visualização é uma etapa crucial na análise de dados, pois permite aos analistas e cientistas de dados entender e interpretar os dados de maneira eficaz, além de comunicar suas descobertas de forma clara. No R, uma linguagem de programação amplamente utilizada para estatística e análise de dados, existem várias ferramentas e pacotes dedicados à visualização de dados. Aqui está uma explanação detalhada sobre os fundamentos da visualização de dados em R.

\subsubsection{Gráficos base do R}\label{gruxe1ficos-base-do-r}

O R possui um sistema de gráficos base que permite a criação de uma ampla variedade de gráficos, incluindo gráficos de linhas, barras, histogramas, boxplots, e scatter plots. Esses gráficos são gerados usando funções simples, como `plot()`, `hist()`, `barplot()`, e `boxplot()`. Por exemplo, a função `plot()` pode ser usada para criar gráficos de dispersão ou linhas, dependendo dos argumentos fornecidos. A função `hist()` é usada para criar histogramas, que são úteis para visualizar a distribuição de uma única variável numérica. Veja, abaixo, uma lista de gráficos comuns em \emph{Base R}:

\begin{enumerate}
\def\labelenumi{\arabic{enumi}.}
\item
  \textbf{Gráficos de dispersão (scatter plots)}: utilizados para visualizar a relação entre duas variáveis numéricas. A função `plot()` é a mais comum para criar esses gráficos.
\item
  \textbf{Histogramas}: usados para representar a distribuição de uma única variável numérica. A função `hist()` é utilizada para gerar histogramas.
\item
  \textbf{Gráficos de linhas}: úteis para visualizar dados ao longo do tempo ou sequências ordenadas. A função `plot()` também pode ser usada para criar gráficos de linhas, alterando o tipo de ponto para uma linha.
\item
  \textbf{Gráficos de barras}: permitem comparar quantidades entre diferentes grupos. As funções `barplot()` ou `hist()` podem ser usadas para criar gráficos de barras.
\item
  \textbf{Boxplots}: fornecem uma representação visual da distribuição de uma variável numérica, destacando a mediana, os quartis e os valores discrepantes. A função `boxplot()` é usada para criar boxplots.
\end{enumerate}

Vamos praticar um exemplo da função ``barplot( )''.

\textbf{Exemplo}: criando um gráfico de barras com a função ``barplot( )''

\includegraphics{images/clipboard-18805364.png}

*Considerando um vetor contendo a contagem de frequências para quatro categorias distintas (Categorias A, B, C e D), um gráfico de barras pode ser gerado com a função ``barplot( )''. Também é possível personalizar diversos aspectos do gráfico dentro da função: ``main'' = título do gráfico; ``xlab'' = título do eixo x; ``ylab'' = título do eixo y; ``col'' = cor de preenchimento das barras; ``ylim'' = limites numéricos do eixo y.

\subsubsection{Personalização de gráficos}\label{personalizauxe7uxe3o-de-gruxe1ficos}

Os gráficos base do R são altamente personalizáveis. Você pode modificar títulos, rótulos dos eixos, cores, tipos de linhas e pontos, e muito mais, usando argumentos adicionais nas funções de plotagem. Por exemplo, você pode usar o argumento `main` para adicionar um título ao gráfico, `xlab` e `ylab` para rótulos dos eixos X e Y, respectivamente, e `col` para alterar a cor dos pontos ou linhas.

Também é possível sobrepor elementos gráficos em R base. Por exemplo, você pode gerar uma linha que conecta todos os pontos em um gráfico de dispersão (``plot( )'' + ``lines( )''), ou até mesmo adicionar uma legenda para diferentes barras (``barplot( )'' + ``legend( )'').

\textbf{Exemplo}: sobrepondo elementos gráficos em R base

\includegraphics{images/clipboard-2859898208.png}

*No exemplo acima, um gráfico de dispersão foi criado utilizando a função ``plot( )''. Logo abaixo, uma função ``lines( )'' cria uma linha vermelha que conecta todos os pontos do gráfico. Finalmente, a função ``legend( )'' permite a criação de uma legenda para os dados.

\subsubsection{O pacote ggplot2}\label{o-pacote-ggplot2}

Além do sistema de gráficos base, o R oferece o pacote `ggplot2` (lembre-se que pacotes devem ser instalados no primeiro uso e importados sempre que utilizados em uma nova sessão), que é baseado nos princípios da ``Gramática dos Gráficos''. O `ggplot2` permite a construção de gráficos complexos de maneira incremental, adicionando camadas de elementos gráficos. Um gráfico no `ggplot2` começa com a função `ggplot()`, à qual são adicionadas camadas usando o operador `+`. Por exemplo, você pode começar definindo os dados e as estéticas (`aes`) e, em seguida, adicionar uma camada para o tipo de gráfico (como `geom\_point()` para um scatter plot ou `geom\_histogram()` para um histograma).

\textbf{Exemplo}: criando um gráfico básico com a função ``ggplot( )''

\includegraphics{images/clipboard-1758074744.png}

*Primeiro, os pacotes desejados foram importados para o ambiente R. No caso do exemplo acima, ``ggplot2'' é o pacote necessário para gerar gráficos, enquanto ``ggpubr'' será utilizado para manipular a apresentação do gráfico no próximo exemplo. Antes de criar o gráfico, um data frame (``df'') foi gerado a partir dos vetores ``Hora'' e ``Temperatura''. Depois, um objeto ggplot é criado utilizando a função ``ggplot(data = df, aes(x = Hora, y = Temperatura))''; para adicionar os pontos ao gráfico, porém, é necessário adicionar um ``+'' ao lado da função e, em seguida, informar a função para o tipo de gráfico. No caso do exemplo, a função ``geom\_point( )'' cria um gráfico de dispersão simples.

Múltiplos gráficos `ggplot` podem ser organizados em uma mesma figura. Uma forma de fazê-lo é com o pacote `ggpubr`. A função `ggarrange()` no R é uma ferramenta flexível do pacote `ggpubr` que permite organizar múltiplos gráficos `ggplot` em uma única página, com a capacidade de especificar o número de colunas e linhas, bem como criar uma legenda comum para vários gráficos.

Por exemplo, se você tiver três gráficos diferentes armazenados nas variáveis `bxp`, `dp` e `dens`, você pode organizá-los em uma grade de 2 colunas e 2 linhas usando `ggarrange(bxp, dp, dens, ncol = 2, nrow = 2)`. Além disso, se você deseja usar uma legenda comum para múltiplos gráficos que compartilham as mesmas estéticas, como cor ou forma, você pode fazer isso com `ggarrange(bxp, dp, common.legend = TRUE)`, onde `bxp` e `dp` são dois gráficos que você deseja combinar. Isso é particularmente útil quando você tem várias visualizações dos mesmos dados e quer evitar a repetição de legendas, tornando o resultado mais limpo e fácil de ler.

\textbf{Exemplo}: organizando mais de um gráfico por tela com ``ggarrange( )''

\includegraphics{images/clipboard-469555772.png}

*Utilizando uma continuação do exemplo anterior, um novo gráfico foi criado (``geom\_histogram( )''). Depois, atribuiu-se o primeiro gráfico à variável ``a'' e o segundo à variável ``b''. Finalmente, bastou informar à função ``ggarrage( )'' quais gráficos devem ser organizados no mesmo plano. Tente inverter a ordem dos objetos na função ``ggarrange'' (ex.: ``ggarrange(b, a)'').

\subsubsection{Visualização de dados multivariados}\label{visualizauxe7uxe3o-de-dados-multivariados}

Tanto o sistema de gráficos base quanto o `ggplot2` oferecem opções para visualizar dados multivariados. Por exemplo, você pode usar cores, formas ou tamanhos de pontos para representar variáveis adicionais em um scatter plot. No `ggplot2`, isso é feito especificando estéticas adicionais dentro da função `aes()`.

A facetação é uma técnica poderosa no `ggplot2` que permite dividir os dados em subconjuntos e plotar cada subconjunto em seu próprio painel dentro do gráfico. Isso é útil para comparar subgrupos de dados. A facetação pode ser realizada usando as funções `facet\_wrap()` ou `facet\_grid()`.

\textbf{Exemplo}: utilizando ``facet\_grid( )'' para facetar um gráfico em seus segmentos

\includegraphics{images/clipboard-928999128.png}

*O código do exemplo acima cria um conjunto de dados e um gráfico correspondente. Primeiramente, são criadas quatro variáveis: `produto`, que repete os nomes ``Produto A'' e ``Produto B'' doze vezes cada; `regiao`, que repete os nomes ``Norte'', ``Sul'' e ``Leste'' quatro vezes cada; `trimestre`, que repete os trimestres ``T1'', ``T2'', ``T3'' e ``T4'' seis vezes; e `vendas`, que contém uma série de valores representando as vendas. Essas variáveis são combinadas em um `data.frame` chamado `dados`. Em seguida, utiliza-se o ggplot2 para criar um gráfico de linhas com pontos, onde o eixo x representa os trimestres, o eixo y as vendas, as linhas e pontos são agrupados e coloridos de acordo com o produto, e o gráfico é dividido em painéis por região utilizando a função ``facet\_grid( )''. Além disso, são definidas cores manuais para os produtos, adicionados títulos aos eixos e ao gráfico, e aplicado um tema minimalista.

\subsubsection{Extensões do ggplot2}\label{extensuxf5es-do-ggplot2}

Existem vários pacotes que estendem as capacidades do `ggplot2`, adicionando novos tipos de gráficos ou funcionalidades. Por exemplo, o pacote `ggplotly` permite converter gráficos `ggplot2` em gráficos interativos, e o `gganimate` permite criar animações a partir de gráficos `ggplot2`.

\textbf{Exemplo}: criando um gráfico interativo a partir de um ggplot com ``ggplotly( )''

\includegraphics{images/clipboard-465180566.png}

*Baseado no código do exemplo anterior, foi possível elaborar um gráfico interativo em três simples passos: 1. Importar o pacote ``plotly'' utilizando ``library(plotly)''; 2. Atribuir o ggplot previamente criado a uma variável ``a'' (basta inserir ``a \textless- `` logo antes do início do código para o gráfico; 3. Fornecer o objeto ``a'' à função ``ggplotly( )''. O gráfico gerado, apesar de similar ao anterior, se torna interativo e permite funções como zoom, salvar como figura, etiqueta dinâmica de dados e outras.

A visualização de dados em R é uma área rica e flexível, com muitas opções disponíveis para o analista. Seja usando o sistema de gráficos base para gráficos simples e rápidos ou explorando a profundidade e flexibilidade do `ggplot2` para gráficos mais complexos e polidos, o R oferece ferramentas poderosas para transformar dados brutos em insights visuais compreensíveis.

\subsubsection{Visualização de dados geoespaciais}\label{visualizauxe7uxe3o-de-dados-geoespaciais}

A visualização de dados geoespaciais é outra técnica avançada que pode ser realizada em R, utilizando pacotes como `ggplot2` em conjunto com `sf` (\citeproc{ref-pebesma2024}{Pebesma et al. 2024}) para criar mapas detalhados. Esses mapas podem representar desde a distribuição geográfica de eventos até análises espaciais complexas, como clusters ou padrões de movimento.

\textbf{Exemplo}: criando um mapa a partir de dados geoespaciais com o pacote ``sf''

\includegraphics{images/clipboard-3449562011.png}

*O código do exemplo acima utiliza as bibliotecas `ggplot2`, `sf` e `rnaturalearth` para criar um mapa do Brasil. A função `ne\_countries` do pacote `rnaturalearth` é usada para obter os dados geográficos do Brasil, especificando que queremos os dados em uma escala grande e que o formato de retorno seja um objeto `sf`, que é um padrão para trabalhar com dados espaciais no R. O objeto `geo\_vetor\_brasil` contém esses dados e é passado para a função `ggplot`, que inicia a construção do gráfico. A função `geom\_sf` é utilizada para adicionar os dados espaciais ao gráfico, com a cor das linhas definida como verde e o preenchimento como amarelo, representando as fronteiras e o interior do país, respectivamente. O tema `theme\_minimal` é aplicado para um visual limpo e a função `labs` adiciona um título ao gráfico. O resultado é um mapa do Brasil com fronteiras destacadas em verde e interior em amarelo.

\subsubsection{Animações com gganimate}\label{animauxe7uxf5es-com-gganimate}

O pacote `gganimate` (\citeproc{ref-pedersen2024}{Pedersen et al. 2024}) estende as capacidades do `ggplot2`, permitindo criar animações a partir de gráficos. Isso pode ser particularmente útil para visualizar mudanças nos dados ao longo do tempo, como tendências econômicas, padrões climáticos ou o progresso de uma epidemia.

\textbf{Exemplo}: criando um gráfico animado com o pacote ``gganimate''

\includegraphics{images/clipboard-342580043.png}

*No exemplo acima, foram utilizadas as bibliotecas `gapminder`(\citeproc{ref-bryan2023}{Bryan 2023}) e `gganimate` para criar uma animação que mostra a relação entre o PIB per capita (GDP per capita), a expectativa de vida (life expectancy) e o tamanho da população (pop) de diferentes países ao longo do tempo. A função `ggplot` inicia a construção do gráfico com os dados do `gapminder`, e a função `aes` define as variáveis para os eixos x e y, o tamanho dos pontos e a cor de acordo com o país. `geom\_point` adiciona pontos ao gráfico com uma transparência de 0.7 e sem legenda. `scale\_colour\_manual` permite a personalização das cores dos pontos por país, enquanto `scale\_size` ajusta o intervalo de tamanho dos pontos. `scale\_x\_log10` transforma o eixo x em uma escala logarítmica para melhor visualização dos dados. `facet\_wrap` divide o gráfico por continente. As funções específicas do `gganimate`, como `transition\_time` e `ease\_aes`, são usadas para criar a animação ao longo do tempo, definida pela variável `year`, e para suavizar a transição entre os frames. O título do gráfico é atualizado dinamicamente para mostrar o ano correspondente a cada frame da animação. Veja a animação \href{https://gganimate.com/reference/figures/README-unnamed-chunk-4-1.gif}{clicando aqui}.

\subsubsection{Dashboards interativos com Shiny}\label{dashboards-interativos-com-shiny}

O pacote `shiny` (\citeproc{ref-chang2024}{Chang et al. 2024}) permite criar aplicativos web interativos diretamente em R. Esses dashboards podem incluir uma variedade de elementos de entrada e saída, permitindo aos usuários interagir com os dados e visualizações em tempo real. Shiny é uma ferramenta poderosa para a construção de ferramentas analíticas interativas e personalizadas.

\textbf{Exemplo}: aplicação shiny para metanálises criada em RStudio

\includegraphics{images/clipboard-3316300773.png}

*A produção de aplicações e \emph{dashboards shiny} é um processo mais complexo do que os exemplos vistos anteriormente. Sua utilização, porém, permite a criação de ferramentas especializadas para pesquisa, trabalho e negócios. Depois de criadas, essas aplicações podem poupar muitas horas de trabalho de um analista ou até mesmo permitir que pessoas sem o conhecimento de programação possam realizar as análises necessárias.

\subsubsection{Tabelas e quadros em R}\label{tabelas-e-quadros-em-r}

Para criar tabelas no R, podemos usar a função `table()` para tabular dados e a função `View()` para visualizar data frames em formato de tabela. Uma forma de visualizar seus dados sem a utilização da função ``View( )'' é clicar no ícone de uma tabela ou uma lupa:

\includegraphics{images/clipboard-4018646742.png}

que aparecem na aba ``Environment'' da janela superior direita. Além disso, o pacote `knitr` (\citeproc{ref-xieaut2024}{Xie {[}aut et al. 2024}) oferece a função `kable()`, que cria tabelas formatadas para relatórios e documentos.

\textbf{Exemplo}: criando uma tabela com a função ``kable( )''

\includegraphics{images/clipboard-2208731083.png}

*Utilizando dados modificados do exemplo anterior, uma tabela kable foi criada. Primeiro, o pacote ``knitr'' foi importado para o ambiente. Em seguida, criou-se um objeto ``table'' com a função ``kable( )'' que recebeu o data frame ``dados''. Finalmente, fornecendo o objeto ``table'' à função ``kable\_classic( )'' (não é necessário utilizar ``kableExtra::'' antes da função), foi possível criar uma tabela.

\section{Python para análise estatística}\label{python-para-anuxe1lise-estatuxedstica}

\section{Visualização de dados e figuras científicas}\label{visualizauxe7uxe3o-de-dados-e-figuras-cientuxedficas}

\chapter{Comunicação científica}\label{comunicauxe7uxe3o-cientuxedfica}

\section{Redação de artigos científicos}\label{redauxe7uxe3o-de-artigos-cientuxedficos}

\section{Estrutura de relatórios de pesquisa}\label{estrutura-de-relatuxf3rios-de-pesquisa}

\section{Apresentações orais e posters}\label{apresentauxe7uxf5es-orais-e-posters}

\section{Publicação e revisão por pares}\label{publicauxe7uxe3o-e-revisuxe3o-por-pares}

\chapter{Ética e boas práticas em pesquisa}\label{uxe9tica-e-boas-pruxe1ticas-em-pesquisa}

\section{Consentimento informado e conformidade ética}\label{consentimento-informado-e-conformidade-uxe9tica}

\section{Plágio e integridade acadêmica}\label{pluxe1gio-e-integridade-acaduxeamica}

\section{Regulamentações e normas em pesquisa}\label{regulamentauxe7uxf5es-e-normas-em-pesquisa}

\section{Responsabilidade social do cientista}\label{responsabilidade-social-do-cientista}

\chapter{Aplicações práticas e estudos de caso}\label{aplicauxe7uxf5es-pruxe1ticas-e-estudos-de-caso}

\section{Epidemiologia e saúde pública}\label{epidemiologia-e-sauxfade-puxfablica}

\section{Genética e biologia molecular}\label{genuxe9tica-e-biologia-molecular}

\section{Ecologia e conservação}\label{ecologia-e-conservauxe7uxe3o}

\section{Farmacologia e ensaios clínicos}\label{farmacologia-e-ensaios-cluxednicos}

\chapter{Tópicos avançados em bioestatística}\label{tuxf3picos-avanuxe7ados-em-bioestatuxedstica}

\section{Modelos lineares generalizados}\label{modelos-lineares-generalizados}

\section{Análise de sobrevivência}\label{anuxe1lise-de-sobrevivuxeancia}

\section{\texorpdfstring{Bioinformática e \emph{big data}}{Bioinformática e big data}}\label{bioinformuxe1tica-e-big-data}

\section{Metanálise}\label{metanuxe1lise}

\chapter{Recursos adicionais}\label{recursos-adicionais}

\section{Bibliografia recomendada}\label{bibliografia-recomendada}

\section{Glossário de termos}\label{glossuxe1rio-de-termos}

\section{Tabelas estatísticas}\label{tabelas-estatuxedsticas}

\section{Links e ferramentas online}\label{links-e-ferramentas-online}

\phantomsection\label{refs}
\begin{CSLReferences}{1}{0}
\bibitem[\citeproctext]{ref-baumer2015}
Baumer, Benjamin, and Dana Udwin. 2015. {``R Markdown.''} \emph{WIREs Computational Statistics} 7 (3): 167--77. \url{https://doi.org/10.1002/wics.1348}.

\bibitem[\citeproctext]{ref-bryan2023}
Bryan, Jennifer. 2023. \emph{Gapminder: Data from Gapminder}. \url{https://cran.r-project.org/web/packages/gapminder/index.html}.

\bibitem[\citeproctext]{ref-chambers1980}
Chambers, John M. 1980. {``Statistical Computing: History and Trends.''} \emph{The American Statistician} 34 (4): 238--43. \url{https://doi.org/10.1080/00031305.1980.10483038}.

\bibitem[\citeproctext]{ref-chang2024}
Chang, Winston, Joe Cheng, J. J. Allaire, Carson Sievert, Barret Schloerke, Yihui Xie, Jeff Allen, et al. 2024. \emph{Shiny: Web Application Framework for r}. \url{https://cran.r-project.org/web/packages/shiny/index.html}.

\bibitem[\citeproctext]{ref-hornik2012}
Hornik, Kurt. 2012. {``The Comprehensive R Archive Network.''} \emph{WIREs Computational Statistics} 4 (4): 394--98. \url{https://doi.org/10.1002/wics.1212}.

\bibitem[\citeproctext]{ref-ihaka1996}
Ihaka, Ross, and Robert Gentleman. 1996. {``R: A Language for Data Analysis and Graphics.''} \emph{Journal of Computational and Graphical Statistics} 5 (3): 299--314. \url{https://doi.org/10.1080/10618600.1996.10474713}.

\bibitem[\citeproctext]{ref-pebesma2024}
Pebesma, Edzer, Roger Bivand, Etienne Racine, Michael Sumner, Ian Cook, Tim Keitt, Robin Lovelace, et al. 2024. \emph{Sf: Simple Features for r}. \url{https://cran.r-project.org/web/packages/sf/index.html}.

\bibitem[\citeproctext]{ref-pedersen2024}
Pedersen, Thomas Lin, David Robinson, Posit, and PBC. 2024. \emph{Gganimate: A Grammar of Animated Graphics}. \url{https://cran.r-project.org/web/packages/gganimate/index.html}.

\bibitem[\citeproctext]{ref-wickham2011}
Wickham, Hadley. 2011. {``Ggplot2.''} \emph{WIREs Computational Statistics} 3 (2): 180--85. \url{https://doi.org/10.1002/wics.147}.

\bibitem[\citeproctext]{ref-wickham2024}
Wickham, Hadley, Peter Danenberg, Gábor Csárdi, Manuel Eugster, Posit Software, and PBC. 2024. \emph{Roxygen2: In-Line Documentation for r}. \url{https://cran.r-project.org/web/packages/roxygen2/index.html}.

\bibitem[\citeproctext]{ref-xieaut2024}
Xie {[}aut, Yihui, cre, Abhraneel Sarma, Adam Vogt, Alastair Andrew, Alex Zvoleff, Amar Al-Zubaidi, et al. 2024. \emph{Knitr: A General-Purpose Package for Dynamic Report Generation in r}. \url{https://cran.r-project.org/web/packages/knitr/index.html}.

\end{CSLReferences}

\end{document}
